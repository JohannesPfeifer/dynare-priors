\documentclass{amsart}

\usepackage[utf8]{inputenc}


\theoremstyle{plain}
\newtheorem{lemma}{Lemma}[section]
\newtheorem{theorem}{Theorem}[section]
\newtheorem{proposition}{Proposition}[section]
\newtheorem{definition}{Definition}[section]

\theoremstyle{remark}
\newtheorem{remark}{Remark}[section]

\numberwithin{equation}{section}

\setlength\parindent{0pt} % No identation in all document.

\begin{document}

\title{Prior distributions in Dynare}
\author{Stéphane Adjemian}
\address{Université du Maine}
\email{stepan@dynare.org}
\date{April, 2016}

\maketitle

\section{Gamma distributions}

\begin{definition}
  The Gamma function is defined as follows:
  \[
    \Gamma(n) = \int_0^{\infty}x^{n-1}e^{-x}\mathrm d x
  \]
  for any $x>0$, zero elsewhere.
\end{definition}

One can easily prove that the foillowing identities hold:
$\Gamma(1) = 1$, $\Gamma\left(\frac{1}{2}\right)=\sqrt{\pi}$ and
$\Gamma(n)=(n-1)\Gamma(n-1)$.

\begin{definition}
  A positive real random variable has a gamma distribution with
  parameters $\alpha>0$ (shape) and $\beta>0$ (scale) if and only of
  its probability density function is given by the following equation:
  \[
    f(x)=\mathcal C(\alpha,\beta)^{-1} \times x^{\alpha-1}e^{-\frac{x}{\beta}}
  \]
  where $\mathcal C(\alpha, \beta) = \Gamma(\alpha)\beta^{\alpha}$ is the constant of
  integration.
\end{definition}
 We will denote $X\sim G(\alpha, \beta)$.  In Dynare this
 distribution may be specified as a prior in the estimated params
 block (using the keyword \verb+GAMMA_PDF+). The user has to specify the
 expectation and standard deviation of the distribution.

 \begin{proposition}
   \label{GammaDistributionMoments}
   If $X\sim G(\alpha, \beta)$, then the expectation and variance of $X$ are:
   \[
     \begin{split}
       \mu &= \alpha\beta \\
       \sigma^2 &= \alpha\beta^2
     \end{split}
   \]
 \end{proposition}

 \begin{proof}[Proof of proposition \ref{GammaDistributionMoments}]
   By definition the expectation is given by:
   \[
     \begin{split}
       \mu &= \mathcal C(\alpha,\beta)^{-1}\int_{0}^{\infty}x^{\alpha}e^{-\frac{x}{\beta}}\mathrm d x \\
       &= \mathcal C(\alpha,\beta)^{-1}\mathcal C(1+\alpha,\beta) \\
       &= \frac{\Gamma(1+\alpha)\beta^{1+\alpha}}{\Gamma(\alpha)\beta^{\alpha}} \\
       &= \frac{\alpha\Gamma(\alpha)\beta^{1+\alpha}}{\Gamma(\alpha)\beta^{\alpha}} \\
       &= \alpha\beta
     \end{split}
   \]
   The second order moment is given by:
   \[
     \begin{split}
       \mathbb E [X^2] &= \mathcal C(\alpha,\beta)^{-1}\int_{0}^{\infty}x^{\alpha+1}e^{-\frac{x}{\beta}}\mathrm d x \\
       &= \mathcal C(\alpha,\beta)^{-1}\mathcal C(2+\alpha,\beta) \\
       &= \frac{(\alpha+1)\alpha\Gamma(\alpha)\beta^{2+\alpha}}{\Gamma(\alpha)\beta^{\alpha}} \\
       &= \alpha(\alpha+1)\beta^2 
     \end{split}
   \]
   and the variance:
   \[
     \mathbb V[X] = \alpha(\alpha+1)\beta^2 - \alpha^2\beta^2 = \alpha\beta^2
   \]
 \end{proof}
 
 \begin{remark}
   Dynare computes the hyperparameters $\alpha$ and $\beta$ by
   inverting the formulas given in proposition
   \ref{GammaDistributionMoments}, we have:
   \[
     \begin{split}
       \alpha &= \frac{\mu^2}{\sigma^2} \\
       \beta &= \frac{\sigma^2}{\mu}
     \end{split}
   \]
 \end{remark}

 \begin{proposition}
   \label{GammaDistributionMode}
   The mode of $X\sim G(\alpha, \beta)$ is:
   \[
     m = 
     \begin{cases}
       0 &\text{if } \alpha\leq 1\text{,} \\
       (\alpha-1)\beta &\text{otherwise.}
     \end{cases}
   \]
 \end{proposition}
\end{document}